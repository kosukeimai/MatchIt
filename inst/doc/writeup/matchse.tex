\documentclass[11pt,titlepage]{article}
%\usepackage[notref]{showkeys}
\usepackage[reqno]{amsmath}
\usepackage{natbib}
\usepackage{amssymb}
\usepackage{epsfig}
\usepackage{comment}
\usepackage{url}
\usepackage[all]{xy}        
\usepackage{dcolumn}
\newcolumntype{.}{D{.}{.}{-1}}
\newcolumntype{d}[1]{D{.}{.}{#1}}
\usepackage{threeparttable,booktabs}
\usepackage{times}
\usepackage{vmargin}
\setpapersize{USletter}
\topmargin=0in
\usepackage[compact]{titlesec}  % small
%\usepackage{ajps}
\usepackage{times}


% Shortcuts
\renewcommand{\P}{\text{P}}
\newcommand{\MC}{\multicolumn}
\usepackage{calc}
\newcounter{hours}\newcounter{minutes}
\newcommand{\printtime}{%
  \setcounter{hours}{\time/60}%
  \setcounter{minutes}{\time-\value{hours}*60}%
  \thehours :\theminutes}
%
\title{The Balance Test Fallacy in Matching Studies\thanks{Our thanks
    to Alberto Abadie, Neal Beck, Alexis Diamond, Guido Imbens, Paul
    Rosenbaum, Don Rubin, and Jas Sekhon for many helpful comments;
    and the National Institutes of Aging (P01 AG17625-01), the
    National Science Foundation (SES-0318275, IIS-9874747,
    SES-0550873), and the Princeton University Committee on Research
    in the Humanities and Social Sciences for research support.}}

\author{Kosuke Imai,\thanks{Assistant Professor, Department of
    Politics, Princeton University (Corwin Hall 041, Department of
    Politics, Princeton University, Princeton NJ 08544, USA;
    \texttt{http://Imai.Princeton.Edu}, \texttt{KImai@Princeton.Edu},
    (609) 258-6601).}
%\and %
  Gary King,\thanks{David Florence Professor of Government, Harvard
    University (Institute for Quantitative Social Science, Harvard
    University, 1737 Cambridge St., Cambridge MA 02138;
    \texttt{http://GKing.Harvard.Edu}, \texttt{King@Harvard.Edu},
    (617) 495-2027).}
%\and %
  Elizabeth A.\ Stuart\thanks{Researcher, Mathematica Policy Research,
    Inc.\, (600 Maryland Ave., SW, Suite 550, Washington, DC, 20024,
    USA; \texttt{http://people.iq.harvard.edu/$\sim$estuart/},
    \texttt{elizabeth.stuart@gmail.com}).}}

\date{\today} 

\begin{document}\maketitle
%\baselineskip=1.57\baselineskip

\begin{abstract}
  In numerous articles across many academic fields, researchers
  evaluate the quality of matching solutions by examining $t-tests$
  for the difference in means between the treatment and control
  groups, or via other hypothesis tests.  We demonstrate that this
  practice is fallacious and misleading.  We also discuss better
  alternatives.
\end{abstract}

A key step in all matching analyses is evaluating the quality of the
matching solution.

  In fact, since hypothesis
tests are driven in part by factors other than balance (including the
number of remaining observations, the ratio of remaining treated to
control units, and the variance of the treated and control groups),
they are not even monotonic functions of balance: the t-test can get
apparently better while balance gets worse, or vice versa.
\section{Introduction}

%\baselineskip=0.637\baselineskip 
\bibliographystyle{apsr}
\bibliography{gk,gkpubs}

\end{document}
